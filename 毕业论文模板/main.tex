%!TEX program = xelatex


% 要使用biber后端
% 要使用xelatex编译
\documentclass[a4paper,12pt]{article}%A4纸,正文为小四号字,对应12pt

% 设置字体
\usepackage{fontspec}
\usepackage[BoldFont,SlantFont,CJKchecksingle]{xeCJK}
\usepackage{ctex}
 

\setCJKmainfont{SimSun}[Path=./fonts/, Extension=.ttf]
\setCJKmonofont{SimSun}[Path=./fonts/, Extension=.ttf]% 设置缺省中文字体为宋体

\usepackage{diagbox}
\usepackage{multido}
\usepackage{float}
\usepackage{algorithmic}
\usepackage{listings}
\usepackage{booktabs}
\usepackage{makecell}
\usepackage{xcolor}



\usepackage[backend=biber,style=gb7714-2015]{biblatex}
\addbibresource{reference.bib} % 替换为你的 .bib 文件


\usepackage{广州大学毕业论文模板}


%%%%%%%%%%%%%%%%%%%%%%%%%%%%%%%

\title{论外语学习与卧榻之地的跨界融合}% 输入论文题目
\school{外国语学院}%学院
%\xibie{系别}%系别
\major{英语} %专业
\class{英语211}%班级
\author{陈清泉}%作者
\studentnumber{20170328}%学号
\supervisor{高育良}%导师姓名

%%%%%%%%%%%%%%%%%%%%%%%%%%%%%%%


\begin{document}

\makecover

\maketitleinfo


\begin{chineseabstract}
	本文以“外语学习”与“休息空间”的跨界融合为研究对象,探讨了在非传统学习环境中语言习得的效率与身心放松的协同效应。通过理论分析与实践案例相结合,本文提出了一种新型的语言学习模式,即在“休息空间”中通过沉浸式体验提升语言能力。研究表明,这种学习方式不仅能有效缓解学习压力,还能通过环境与心理的互动增强语言记忆与应用能力。本文旨在为语言学习领域提供一种创新思路,同时为相关实践提供理论支持。
\end{chineseabstract}



\begin{chinesekeywords}
	外语学习;卧榻之地;跨界融合;语言习得;协同效应;沉浸式体验
\end{chinesekeywords}



\begin{englishabstract}
	This paper focuses on the cross-border integration of "foreign language learning" and "rest space," exploring the efficiency of language acquisition and the synergistic effects of physical and mental relaxation in non-traditional learning environments. Through a combination of theoretical analysis and practical cases, this paper proposes a new language learning model that enhances language proficiency through immersive experiences in "rest spaces." Research shows that this learning method not only effectively alleviates learning stress but also strengthens language memory and application abilities through the interaction of environment and psychology. This paper aims to provide an innovative perspective for the field of language learning while offering theoretical support for related practices.
\end{englishabstract}


\begin{englishkeywords}
	foreign language learning; the land of bed; cross-border integration; language acquisition; synergistic effects; immersive experience
\end{englishkeywords}


\newpage

\tableofcontents % 生成目录
\newpage


\section{前\hspace{1em}言}

语言学习作为一项复杂而多维的认知活动,长期以来一直是教育学、心理学和语言学领域的研究热点。传统的语言学习模式通常依赖于课堂、教材和教师指导,然而,随着学习理论的不断发展,越来越多的研究表明,学习环境对语言习得的效率有着深远的影响。近年来,一种新兴的语言学习模式逐渐引起学界关注——即在非传统学习环境中,通过身心放松与沉浸式体验相结合的方式,提升语言能力。本文正是基于这一背景,探讨“外语学习”与“卧榻之地”的跨界融合,试图为语言学习领域提供一种全新的视角与实践路径。

“卧榻之地”作为一种特殊的物理空间,不仅是人们日常休息与放松的场所,更因其独特的环境属性而具备了潜在的学习功能。本文提出,在“卧榻之地”中进行外语学习,不仅能够有效缓解学习者的心理压力,还能通过环境与心理的互动,增强语言记忆与应用能力。这种学习模式的提出,既是对传统语言学习方法的补充,也是对学习环境理论的进一步拓展。

本文首先从理论层面分析“卧榻之地”作为学习环境的独特优势,随后结合实践案例,探讨其在语言习得中的实际效果。最后,本文将对这种跨界融合的协同效应进行总结,并提出未来研究的方向与建议。通过本研究,我们希望能够为语言学习者提供一种更加轻松、高效的学习方式,同时也为相关领域的理论研究与实践探索提供新的思路。

如图\ref{fig:learn_english}所示,作者本人在“卧榻之地”中进行外语学习的实践场景,生动展示了非传统学习环境对语言习得的潜在影响。通过这种沉浸式体验,学习者能够在身心放松的状态下,更高效地掌握语言技能。

\begin{figure}[htbp]
	\centering
	\includegraphics[width=0.8\textwidth]{./imgs/外语学习证明.jpg} % 图片路径
	\caption{作者本人“外语学习”实践场景:卧榻之地与语言习得的跨界融合\ 图源\cite{A橘色的海2025(补)假如陈清泉真的在学外语}} % 图片标题
	\label{fig:learn_english} % 图片标签
\end{figure}

\input{chapter01.tex}

\input{chapter02}

\input{结论}

\printbibliography[heading=bibintoc]
\input{致谢.tex}

\end{document}