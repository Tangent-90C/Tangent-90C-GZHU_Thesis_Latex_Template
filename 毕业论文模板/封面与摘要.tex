\makecover
\newpage
\maketitleinfo

\begin{chineseabstract}
	本文以“外语学习”与“休息空间”的跨界融合为研究对象,探讨了在非传统学习环境中语言习得的效率与身心放松的协同效应。通过理论分析与实践案例相结合,本文提出了一种新型的语言学习模式,即在“休息空间”中通过沉浸式体验提升语言能力。研究表明,这种学习方式不仅能有效缓解学习压力,还能通过环境与心理的互动增强语言记忆与应用能力。本文旨在为语言学习领域提供一种创新思路,同时为相关实践提供理论支持。
\end{chineseabstract}



\begin{chinesekeywords}
	外语学习;卧榻之地;跨界融合;语言习得;协同效应;沉浸式体验
\end{chinesekeywords}



\begin{englishabstract}
	This paper focuses on the cross-border integration of "foreign language learning" and "rest space," exploring the efficiency of language acquisition and the synergistic effects of physical and mental relaxation in non-traditional learning environments. Through a combination of theoretical analysis and practical cases, this paper proposes a new language learning model that enhances language proficiency through immersive experiences in "rest spaces." Research shows that this learning method not only effectively alleviates learning stress but also strengthens language memory and application abilities through the interaction of environment and psychology. This paper aims to provide an innovative perspective for the field of language learning while offering theoretical support for related practices.
\end{englishabstract}


\begin{englishkeywords}
	foreign language learning; the land of bed; cross-border integration; language acquisition; synergistic effects; immersive experience
\end{englishkeywords}


\newpage
